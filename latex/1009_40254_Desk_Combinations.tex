\documentclass[12pt]{article}
\usepackage[margin=0.5in]{geometry}
\usepackage{siunitx}
\usepackage{booktabs}

\renewcommand{\arraystretch}{1.3}


\begin{document}
\thispagestyle{empty}

\begin{center}
	{\huge\textbf{\underline{ Desk and Locker Combination Key}}}
\end{center}


\begin{table}[h]
  \centering
  \begin{tabular}{ccccc}

  \textbf{Class} & 1009 & {\qquad} &\textbf{Schedule} & Tuesday 3:00pm - 5:00pm \\
  \textbf{Section} & 40254 & {\qquad} & \textbf{Semester} & Fall 2015 \\
  \textbf{Lab} & 233 & {\qquad} & \textbf{Instructor} & Andrew Cox \\
  \end{tabular}
\end{table}
 \vspace{0.5in}
\begin{minipage}{0.4\textwidth}

\begin{tabular}{rrl}
\toprule
 Desk &      ID &     Combo \\
\midrule
   11 &  702838 &  06-16-34 \\
   23 &  702827 &  38-04-10 \\
   41 &  702840 &  24-14-04 \\
   53 &  702829 &  14-32-10 \\
   71 &  702842 &  16-18-08 \\
   83 &  702831 &  10-12-30 \\
  101 &  702844 &  08-10-00 \\
  113 &  702833 &  30-36-14 \\
  131 &  702846 &  08-26-36 \\
  143 &  702835 &  25-27-05 \\
  161 &  702848 &  32-02-12 \\
  173 &  702837 &  30-00-26 \\
\bottomrule
\end{tabular}


\end{minipage}
\begin{minipage}{0.4\textwidth}
\underline{{\large \textbf{Master Lock Combination Instructions}}}
\begin{enumerate}
\item Copy your desk number and lock combination to a secure location that you will have access to.
\item Turn the dial three times to the right, then stop when the first number lines up with the indicator.
\item Turn the dial to the left one full turn (the first number in the combination passes the arrow), continue turning until the second number in the combination is beneath the arrow.
\item Turn the dial to the right, stopping when the third number in the combination is beneath the arrow.
\item Pull the lock shackle to open.
\end{enumerate}
\end{minipage}
\end{document}